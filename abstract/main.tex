\documentclass{article}
\usepackage[margin=1in]{geometry}

\begin{document}

Natural phenomena is often complex and challenging to model. However, attempts
have been to capture such phenomena via a set of simple local rules. For
example, consider the Boids algorithm capable of simulating the movement of flocks
of birds and schools of fish. We hypothesise that emergent behavior is present
in Western honey bees (\textit{Apis mellifera}) as well. In the same spirit, we
constructed a two-dimensional browser-based simulation that demonstrated the
self-assembly of honey bees. Parameterised by two variables, a \textit{crawl rate}
(how fast bees attached to the wooden board advance towards the queen) and
a \textit{climb rate} (how vigorously bees at the peripheral of the structure attempt
to scale the inverted mount), we were able to yield structures akin to those
recorded in the physical world. Quantitatively, our simulation demonstrates
similar results to those established in the literature. Interestingly, our
simulation suggests that at lower-density regions, a lattice of diamonds
is the prevalent structure, while at higher-density regions, this lattice
breaks down into an amorphous mesh of bees with a higher degree of connectivity.

\end{document}
