\documentclass{article}
\usepackage[margin=1in]{geometry}

\begin{document}

Self-assembly pervades nature, from the coalescence of iron oxide nanoparticles
to the collective motion of flocks of birds. To understand self-assembly, we
devised a set of simple local rules to simulate the swarming of Western
honey bees (\textit{Apis mellifera}), thereby giving rise to a spectrum of
interesting emergent states characterised by their morphology and behavior.
Parameterised by two variables, a \textit{crawl rate} (how fast bees attached
to the wooden board advance towards the queen) and a \textit{climb rate} (how
vigorously bees at the peripheral of the structure attempt to scale the inverted
mount), we were able to yield structures akin to those recorded in the
physical world. These findings were then corroborated with empirical data established
in the literature. Our simulation reveals that at lower-density
regions, a diagonally-centered square lattice is the prevalent structure, while
at higher-density regions, this lattice breaks down into an amorphous mesh
of bees with a higher degree of connectivity.

\end{document}
